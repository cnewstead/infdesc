\section{Structural induction}
\label{secStructuralInduction}
\index{induction!structural|(}

In \Cref{secPeanosAxioms}, we formalised the idea that the set of natural numbers should be what is obtained by starting with zero and repeating the successor (`plus one') operation---this was done using Peano's axioms (\Cref{defNotionOfNaturalNumbers}). From these axioms we were able to derive the weak and strong induction principles, which turned out to be extremely powerful for proving results about natural numbers.

We now generalise this idea to other so-called \textit{inductively defined} sets. The definition (\Cref{defInductivelyDefinedSet}) is a little tricky to digest, but the idea is relatively simple: an inductively defined set $X$ is one whose elements are built out of some specified \textit{basic elements} (such as $0$) by iterating some specified operations, called \textit{constructors} (such as the successor operation)---every element of $X$ should either be a basic element, or should be built in a unique way out of simpler elements of $X$ by using a constructor.

Each inductively defined set $X$ will have its own induction principle: which says that if a property is true of all of the basic elements of $X$, and if all of its constructors preserve the truth of the property, then the property is true of all of the elements of $X$. We will prove this in \Cref{thmStructuralInduction}.

Before jumping into the definition of an inductively defined set, it is helpful to see some examples. The first example is familiar.

\begin{example}
\label{exNaturalNumbersAsInductivelyDefinedSetPreliminary}
The set $\mathbb{N}$ of natural numbers is built from the natural number $0$ by adding one. The operation of `adding one' yields the successor operation $s : \mathbb{N} \to \mathbb{N}$, defined by $s(n)=n+1$ for all $n \in \mathbb{N}$. The Peano axioms tell us that every natural number is either $0$, or is equal to $s(n)$ for a unique $n \in \mathbb{N}$.
\end{example}

The next example concerns words over an alphabet, which we studied in \Cref{secCountableUncountableSets} in the context of determining when a set is countable---see \Cref{defKleeneStar}.

\begin{example}
\label{exWordsAsInductivelyDefinedSetPreliminary}
Given an alphabet $\Sigma$, the set $\Sigma^*$ of words over $\Sigma$ is built from the empty word $\varepsilon$ by appending elements $\sigma \in \Sigma$. Each $a \in \Sigma$ induces a function $\sigma_a : \Sigma^* \to \Sigma^*$ defined by $\sigma_a(w) = w\sigma$ for all $w \in \Sigma^*$, and every element of $\Sigma^*$ is either the empty word $\varepsilon$, or is equal to $\sigma_a(w)$ for a unique $a \in \Sigma$ and a unique $w \in \Sigma^*$. 
\end{example}

We alluded to the inductive character of propositional formulae in \Cref{secPropositionalLogic}. Although we were not able to make the idea precise at the time, we are now well on our way to doing so.

\begin{example}
\label{exPropositionalFormulaeAsInductivelyDefinedSetPreliminary}
Given a set $P$, consider the set $L(P)$ of all propositional formulae with propositional variables from $P$ and logical operators from the set $\{ {\wedge}, {\vee}, {\Rightarrow}, {\neg} \}$. Then $L(P)$ is built out of the elements of $P$ using these logical operators.

Each logical operator induces an operation on the set $L(P)$. Indeed, ach logical operator $\odot \in \{ {\wedge}, {\vee}, {\Rightarrow} \}$ induces a function $\sigma_{\odot} : L(P)^2 \to L(P)$ defined by $\sigma_{\odot}(\varphi, \psi) = \varphi \odot \psi$ for all $\varphi,\psi \in L(P)$; and the logical operator $\neg$ induces a function $\sigma_{\neg} : L(P) \to L(P)$ defined by $\sigma_{\neg}(\varphi) = \neg \varphi$ for all $\varphi \in L(P)$.

Every propositional formula in $L(P)$ is then either a propositional variable $p \in P$, or is $\sigma_{\neg}(\varphi)$ for a unique $\varphi \in L(P)$, or is $\sigma_{\odot}(\varphi,\psi)$ for a unique logical operator $\odot \in \{ {\wedge}, {\vee}, {\Rightarrow}, {\neg} \}$ and unique $\varphi,\psi \in L(P)$ .

For example, let $\varphi = (p \wedge q) \Rightarrow (\neg r)$. Then $\varphi = \sigma_{\Rightarrow}( p \wedge q, \neg r )$, and this is the unique expression of $\varphi$ as a constructor applied to simpler propositional formulae. Continuing, we eventually reach the level of basic elements, which are propositional variables in this case: $p \wedge q = \sigma_{\wedge}(p,q)$ and $\neg r = \sigma_{\neg}(r)$.
\end{example}

Keeping these examples in mind, we are now ready to define the notion of an inductively defined set.

\begin{definition}
\label{defInductivelyDefinedSet}
\index{inductively defined set}
\index{set!inductively defined}
\index{constructor}
\index{arity}
\nindex{arity}{$\mathrm{ar}(f)$}{arity of a constructor}
An \textbf{inductively defined set} is a set $A$ together with a set $C_A$ of \textbf{constructors}, which are functions $\sigma : A^r \to A$ for various $r \in \mathbb{N}$, such that
\begin{enumerate}[(i)]
\item For each $a \in A$, there is a unique constructor $\sigma : A^r \to A$ and unique elements $a_1,a_2,\dots,a_r \in A$ such that $a = \sigma(a_1,a_2,\dots,a_r)$; and
\item For all sets $X$, if $\sigma(a_1,a_2,\dots,a_r) \in A$ for all constructors $\sigma : A^r \to A$ and all $a_1,a_2,\dots,a_r \in A$, then $A \subseteq X$.
\end{enumerate}
For a constructor $\sigma : A^r \to A$, the natural number $r$ such that $\sigma : A^r \to A$ is called the \textbf{arity} of $\sigma$, also written as $\mathrm{ar}(\sigma)$ \inlatex{mathrm\{ar\}}.
\end{definition}

A quick note on terminology: a constructor of arity $r \in \mathbb{N}$ is called an \textit{$r$-ary} constructor. For $r=0,1,2$, we may say \textit{nullary}, \textit{unary} and \textit{binary}, respectively.

You might be wondering, where are the basic elements? It turns out that the basic elements are exactly the nullary constructors---after all, if basic elements are simply declared to exist, then they are constructed out of nothing. To make this idea precise, note that $A^0 = \{ () \}$, where $()$ is the empty list of elements of $A$; a function $\sigma : A^0 \to A$ is therefore determined by the value $\sigma(\,()\,) \in A$. Under this identification, we may regard $\sigma$ as \textit{being} an element of $A$!

\begin{definition}
\label{defBasicElement}
\index{basic element}
\index{element!basic}
A \textbf{basic element} of an inductively defined set $A$ is a nullary constructor $\sigma : A^0 \to A$, regarded as an element $\sigma = \sigma(\,()\,) \in A$.
\end{definition}

Let's unpack these definitions in terms of \Crefrange{exNaturalNumbersAsInductivelyDefinedSetPreliminary}{exPropositionalFormulaeAsInductivelyDefinedSetPreliminary}.

\begin{example}
\index{natural number}
The set $\mathbb{N}$ of natural numbers is an inductively defined set. The set $C_{\mathbb{N}}$ of constructors is given by $C_{\mathbb{N}} = \{ 0, s \}$, where $0 \in \mathbb{N}$ is a basic element, and $s : \mathbb{N} \to \mathbb{N}$ a unary constructor defined by $s(n) = n+1$ for all $n \in \mathbb{N}$.

In fact, the conditions of \Cref{defInductivelyDefinedSet} are essentially just restatements of the Peano axioms (\Cref{defNotionOfNaturalNumbers}):
\begin{enumerate}[(i)]
\item Let $n \in \mathbb{N}$.
\begin{itemize}
\item If $n = 0$, then $n \ne s(m)$ for any $m \in \mathbb{N}$ by \Cref{defNotionOfNaturalNumbers}(i), so `$0$' is the unique expression for $0$ as a constructor applied to elements of $\mathbb{N}$.
\item If $n > 0$, then $n-1 \in \mathbb{N}$ and $n=s(n-1)$. To see that this expression is unique, note that if $m \in \mathbb{N}$ and $s(m) = n$, then $m = n-1$ by \Cref{defNotionOfNaturalNumbers}(ii).
\end{itemize}
\item We need to prove that if $X$ is a set such that $0 \in X$ and $s(n) \in X$ for all $n \in \mathbb{N}$, then $\mathbb{N} \subseteq X$. But this is exactly \Cref{defNotionOfNaturalNumbers}(iii).
\end{enumerate}

Hence $\mathbb{N}$ is indeed an inductively defined set.
\end{example}

\begin{example}
The set $\Sigma^*$ of words over an alphabet $\Sigma$ is an inductively defined set. The set of constructors is given by
\[ C_{\Sigma^*} = \{ \varepsilon \} \cup \{ \sigma_a \mid a \in A \} \]
where $\varepsilon \in \Sigma^*$ is a basic element, and for each $a \in A$, we have a unary constructor $\sigma_a : \Sigma^* \to \Sigma^*$ defined by $\sigma_a(w) = wa$ for all $w \in \Sigma^*$.

% To verify conditions (i) and (ii) of \Cref{defInductivelyDefinedSet}, first recall that $\Sigma^* = \bigcup_{n \in \mathbb{N}} \Sigma^n$, where the sets $\Sigma^n$ are defined by
% \[ \Sigma^0 = \{ \varepsilon \} \quad \text{and} \quad \Sigma^{n+1} = \{ wa \mid w \in \Sigma^n,~ a \in \Sigma \} \]
% Thus the elements of $\Sigma^n$ are the words of length $n \in \mathbb{N}$. Now:
% \begin{enumerate}[(i)]
% \item Let $w \in \Sigma^*$.
% \begin{itemize}
% \item If $w \in \Sigma^0$, then $w = \varepsilon$, and therefore $w \ne w'a$ for any $a \in \Sigma$.
% \item If $w$ has length $n+1$ for some $n \in \mathbb{N}$, then $w = w'a$ for some $w' \in \Sigma^n$ and $a \in \Sigma$. This expression is unique: if $w = w''b$ for some $w'' \in \Sigma^*$ and $b \in \Sigma$, then $w'' \in \Sigma^n$ since $w$ must have length $n+1$; and $a=b$ since the last character of two equal words must be the same; and then $w'a=w''a$, so that $w'=w''$, since deleting the last character from two equal words results in the same word. So $w=\sigma_a(w')$ is the unique expression of $w$ as a constructor applied to an element of $\Sigma^*$.
% \end{itemize}

% \item Let $X$ be a set and suppose that $\varepsilon \in X$ and, for all $w \in \Sigma^*$, if $w \in X$ then $\sigma_a(w) \in X$ for all $a \in \sigma$. To see that $\Sigma^* \subseteq X$, let $w \in \Sigma^*$. We prove by induction that $\Sigma^n \subseteq X$ for all $n \in \mathbb{N}$.
% \begin{itemize}
% \item (\textbf{Base case}) Let $w \in \Sigma^0$. Then $w=\varepsilon \in X$, so that $\Sigma^0 \subseteq X$.
% \item (\textbf{Induction step}) Let $n \in \mathbb{N}$, assume that $\Sigma^n \subseteq X$, and let $w \in \Sigma^{n+1}$. Then $w = w'a = \sigma_a(w')$ for some $w' \in \Sigma^n$. But then $w' \in X$ by the induction hypothesis, so that $w = \sigma_a(w') \in X$. Hence $\Sigma^{n+1} \subseteq X$.
% \end{itemize}
% It follows that $\Sigma^* = \bigcup_{n \in \mathbb{N}} \Sigma^n \subseteq X$, as required.
% \end{enumerate}
\end{example}

\begin{example}
Let $P$ be a set of propositional variables. Describe the set $L(P)$ of propositional formulae (as in \Cref{exPropositionalFormulaeAsInductivelyDefinedSetPreliminary}) as an inductively defined set.
\end{example}

\begin{exercise}
Prove that the set $A = \{ 1, 2, 4, 8, 16 \dots \}$ of (natural number) powers of $2$ is inductively defined by taking $C_A = \{ 1, d \}$, where $1$ is basic and $d : A \to A$ is defined by $d(n) = 2n$ for all $n \in \mathbb{N}$.
\end{exercise}

The next exercise gives a different way of inductively defining $\mathbb{N}$---it demonstrates that we can consider a set to be inductively defined in more than one way.

\begin{exercise}
Prove that $\mathbb{N}$ is inductively defined by taking $C_{\mathbb{N}} = \{ 1, \sigma \}$, where $\sigma : \mathbb{N} \to \mathbb{N}$ is defined by
\[ \sigma(n) = \begin{cases} n-1 & \text{if $n$ is odd} \\ n+3 & \text{if $n$ is even} \end{cases} \]
for all $n \in \mathbb{N}$.
\end{exercise}

So far we have seen how to describe sets that have \textit{already been defined} as inductively defined sets. In practice, it is much more convenient to be able to make an inductive definition of a set by describing how to construct its elements.

\begin{definition}
\label{defRuleForInductiveDefinition}
A (\textbf{finitary}) \textbf{rule} for an inductive definition is an expression of the form
\begin{center}
\begin{prooftree}
    \AxiomC{$x_1$}
    \AxiomC{$x_2$}
    \AxiomC{$\cdots$}
    \AxiomC{$x_r$}
\QuaternaryInfC{$\sigma(x_1,x_2,\dots,x_r)$}
\end{prooftree}
\end{center}
where $r \in \mathbb{N}$, $x_1,x_2,\dots,x_r$ are variables, and $\sigma(x_1,x_2,\dots,x_r)$ is some expression involving the variables $x_1,x_2,\dots,x_r$. The natural number $r$ is called the \textbf{arity} of the rule.
\end{definition}

Rules describe how an inductively defined set $(A, C_A)$ is built up; the rule in \Cref{defRuleForInductiveDefinition} can be read as saying that if $x_1, x_2, \dots, x_r \in A$, then $\sigma(x_1,x_2,\dots,x_r) \in A$. Note that if $r=0$, then this amounts to declaring a basic element of $A$.

\todo{}

\begin{center}
\begin{tikzpicture}
\draw (-2, 2) node(x1) {$x_1$} ;
\draw (-1, 2) node(x2) {$x_2$} ;
\draw (0.5, 2) node(dots) {$\cdots$} ;
\draw (2, 2) node(xn) {$x_n$} ;
  \draw (0, 1) \sdconstructor{sigma}{$\sigma$} ;
    \draw (0, 0) node(result) {$\sigma(x_1,x_2,\dots,x_n)$} ;
\draw[-latex] (x1) -- (sigma) -- (result) ;
\draw (x2) -- (sigma) ;
\draw (xn) -- (sigma) ;
\end{tikzpicture}
\end{center}

We can then piece these rules together

\todo{}

\begin{example}
The rules defining the natural numbers are
\begin{center}
\begin{minipage}{0.2\textwidth}
\centering
\begin{prooftree}
    \AxiomC{~}
\UnaryInfC{$0$}
\end{prooftree}
\end{minipage}
%
\text{ and }
%
\begin{minipage}{0.2\textwidth}
\centering
\begin{prooftree}
    \AxiomC{$n$}
\UnaryInfC{$s(n)$}
\end{prooftree}
\end{minipage}
\end{center}
The first of these rules says that $0$ is a natural number; the second says that if $n$ is a natural number, then its successor $s(n)$ is a natural number.
\end{example}

\begin{example}
Let $P$ be a set of propositional variables. The rules defining the set $L(P)$ of propositional formulae over $P$ are given by
\begin{center}
\begin{minipage}{0.19\textwidth}
\centering
\begin{prooftree}
    \AxiomC{~}
\TagC{$\large p \in P$}
\UnaryInfC{$p$}
\end{prooftree}
\end{minipage}
%
\hfill
%
\begin{minipage}{0.19\textwidth}
\centering
\begin{prooftree}
    \AxiomC{$\varphi$}
    \AxiomC{$\psi$}
\BinaryInfC{$\varphi \wedge \psi$}
\end{prooftree}
\end{minipage}
%
\hfill
%
\begin{minipage}{0.19\textwidth}
\centering
\begin{prooftree}
    \AxiomC{$\varphi$}
    \AxiomC{$\psi$}
\BinaryInfC{$\varphi \vee \psi$}
\end{prooftree}
\end{minipage}
%
\hfill
%
\begin{minipage}{0.19\textwidth}
\centering
\begin{prooftree}
    \AxiomC{$\varphi$}
    \AxiomC{$\psi$}
\BinaryInfC{$\varphi \Rightarrow \psi$}
\end{prooftree}
\end{minipage}
%
\hfill
%
\begin{minipage}{0.19\textwidth}
\centering
\begin{prooftree}
    \AxiomC{$\varphi$}
\UnaryInfC{$\neg \varphi$}
\end{prooftree}
\end{minipage}
\end{center}
The first of these rules says that every propositional variable $p \in P$ is a propositional formula. The next three say that if $\varphi$ and $\psi$ are propositional formulae, then so are $\varphi \wedge \psi$, $\varphi \vee \psi$ and $\varphi \Rightarrow \psi$. The last rule says that if $\varphi$ is a propositional formula, then so is $\neg \varphi$.
\end{example}

\begin{exercise}
Write rules for inductively defining the set $\Sigma^*$ of words over an alphabet $\Sigma$.
\end{exercise}

\begin{definition}
\label{defSetGeneratedByRules}
Let $R$ be a set of rules for an inductive definition. The set \textbf{generated by} $R$ is defined by $A = \bigcup_{n \in \mathbb{N}} A_n$, where the sets $A_n$ for $n \in \mathbb{N}$ are defined recursively by:
\[ A_0 = \left\{ a \middlemid \dfrac{\hspace{20pt}}{a} \in R \right\} \]
and
\[ A_{n+1} = \left\{ \sigma(a_1,a_2,\dots,a_r) \middlemid a_1,a_2,\dots,a_r \in A_n, \quad \dfrac{x_1 \hspace{10pt} x_2 \hspace{10pt} \cdots \hspace{10pt} x_r}{\sigma(x_1,x_2,\dots,x_r)} \in R \right\} \]
That is, $A_0$ is the set of basic elements declared to exist by the nullary rules in $R$, and $A_{n+1}$ is the set of all expressions obtained by applying the rules in $R$ to the elements of $A_n$.
\end{definition}

\todo{}

\begin{exercise}
Let $R$ be a set of rules for an inductive definition and let $A$ be the set generated by $R$. Prove that if $R$ has no nullary rules, then $A$ is empty.
\end{exercise}

\begin{exercise}
Let $R$ be a set of rules for an inductive definition, and let $A$ be the set generated by $R$. Prove that if $R$ is countable, then $A$ is countable.
\end{exercise}

\todo{}

\begin{definition}
\label{defRank}
\index{rank}
\nindex{rk}{$\mathrm{rk}(a)$}{rank}
Let $(A, C_A)$ be an inductively defined set. The \textbf{rank} of an element $a \in A$ is the (unique) natural number $n$ such that $a \in A_n$, also written $\mathrm{rk}(a)$ \inlatex{mathrm\{rk\}}.
\end{definition}

\todo{}

\begin{theorem}
Let $R$ be a set of rules for an inductive definition, and let $A$ be the set generated by $R$. Then $A$ is an inductively defined set, with each $r$-ary rule $\dfrac{x_1 \hspace{10pt} x_2 \hspace{10pt} \cdots \hspace{10pt} x_r}{\sigma(x_1,x_2,\dots,x_r)}$ defining an $r$-ary constructor $\sigma : A^r \to A$ in the evident way.
\end{theorem}

\begin{cproof}
We must verify conditions (i) and (ii) of \Cref{defInductivelyDefinedSet}.
\begin{enumerate}[(i)]
\item Let $a \in A$. We prove by induction on $n \in \mathbb{N}$ that, if $a \in A_n$, then $a = \sigma(a_1,a_2,\dots,a_r)$ for a unique constructor $\sigma : A^r \to A$ and unique $a_1,a_2,\dots,a_r \in A$, and moreover $a_1,a_2,\dots,a_r \in A_k$ for some $k < n$.

\begin{itemize}
\item (\textbf{Base case}) Suppose $a \in A_0$. Then $a$ is a basic element of $A$, corresponding with the nullary rule $\dfrac{\hspace{20pt}}{a}$, so that $a$ is already a nullary constructor. But then $a$ involves no variables, so that $a$ does not arise from an $r$-ary rule for $r \ge 1$---hence if $a=\sigma(a_1,a_2,\dots,a_r)$ for some constructor $\sigma : A^r \to A$, then $r=0$ and $a = \sigma(\,()\,)$; but this says precisely that `$a$' is the unique expression of $a$ as a constructor applied to some elements of $A$.

\item (\textbf{Induction step}) Fix $n \ge 0$ and assume that, for all $k \le n$, if $b \in A_k$, then $b = \sigma(b_1,b_2,\dots,b_r)$ for a unique constructor $\sigma : A^r \to A$ and unique elements $b_1,b_2,\dots,b_r \in A$.

\end{itemize}

\item \todo{}
\end{enumerate}
\end{cproof}

\todo{}

\subsection*{Structural induction}

\todo{}

\begin{theorem}[Structural induction principle]
\label{thmStructuralInduction}
\index{induction!structural}
Let $(A, C_A)$ be an inductively defined set, and let $p(x)$ be a logical formula with free variable $x \in A$.

If for all constructors $\sigma : A^r \to A$ and all $a_1,\dots,a_r \in A$ we have
\[ (\forall i \in [r],~p(a_i)) \Rightarrow p(\sigma(a_1,a_2,\dots,a_r)) \]
then $p(x)$ is true for all $x \in A$.
\end{theorem}

\begin{cproof}
Let $a \in A$. We prove that $p(a)$ is true by strong induction on the rank of $a$.
\begin{itemize}
\item (\textbf{Base case}) Suppose $\mathrm{rk}(a) = 0$. Then $a = \sigma(\,()\,)$ for some nullary constructor $\sigma$, and so the assumption in the statement of the theorem says that
\[ (\forall i \in [0],~p(a_i)) \Rightarrow p(a) \]
The hypothesis $\forall i \in [0],~p(a_i)$ is satisfied vacuously, since $[0] = \varnothing$, and so $p(a)$ is true.

\item (\textbf{Induction step}) Let $n \in \mathbb{N}$ and suppose that $p(a)$ is true for all $a \in A$ with $\mathrm{rk}(a) \le n$. Let $a \in A$ and assume that $\mathrm{rk}(a) = n+1$; we need to prove that $p(a)$ is true.

Since $\mathrm{rk}(a) = n+1$, it follows that $a = \sigma(a_1,a_2,\dots,a_r)$ for some constructor $\sigma$ and some $a_1,a_2,\dots,a_r \in A$ with $\mathrm{rk}(a_i) \le n$ for all $i \in [r]$.

By the induction hypothesis, $p(a_i)$ is true for all $i \in [r]$. By the assumption in the statement of the theorem, it follows that $p(\sigma(a_1,a_2,\dots,a_r))$ is true. But $\sigma(a_1,a_2,\dots,a_r) = a$, so that $p(a)$ is true.
\end{itemize}

By induction, it follows for all $n \in \mathbb{N}$ that $p(a)$ is true for all $a \in A_n$. But then $p(a)$ is true for all $a \in A$.
\end{cproof}

\todo{}

\begin{example}

\end{example}

\todo{}

\begin{definition}
\todo{Ancestor relation $<_A$} 
\end{definition}

\todo{}

\begin{theorem}[Strong structural induction principle]
Let $(A, C_A)$ be an inductively defined set, and let $p(x)$ be a logical formula with free variable $x \in A$.

If for all $a \in A$ we have
\[ (\forall b <_A a,~ p(b)) \Rightarrow p(a) \]
then $p(x)$ is true for all $x \in A$.
\end{theorem}

\begin{cproof}
Let $q(x)$ be the logical formula defined by $\forall b <_a x,~ p(b)$. We prove $\forall x \in A,~ q(x)$ by structural induction on $x$.
\end{cproof}


\textbf{--- OLD CONTENT BEGINS HERE ---}

We will prove \Cref{thmStructuralInduction} on page \pageref{prfStructuralInduction}.

\begin{example}
\label{exStructuralInductoinOnNIsWeakInduction}
\todo{Structural induction on $\mathbb{N}$ is weak induction.}
\end{example}

\todo{Disjunctive normal form}

\todo{Generalise to quotients of inductive structures $\leadsto$ induction on $\mathbb{Z}$ using $0$ and $+,-$ and on $\mathbb{Z}^{>0}$ using $1$ and $p \times (-)$.}

We saw in \Cref{propPositiveIntegersAsWellFoundedSet} that the relation $R$ on the set $\mathbb{Z}^{>0}$ of positive integers defined for $m,n \in \mathbb{Z}^{>0}$ by
\[ m\; R\; n \quad \Leftrightarrow \quad n=pm \text{ for some prime } p>0 \]
is well-founded. We can use well-founded induction to prove a general formula for the totient of an integer $n$.

\begin{theorem}[Formula for Euler's totient function]
\label{thmFormulaForTotientFunction}
Let $n \in \mathbb{Z}$ be nonzero, and let $\varphi : \mathbb{Z} \to \mathbb{N}$ be Euler's totient function (see \Cref{defTotient}). Then
\[ \varphi(n) = |n| \cdot \prod_{p \mid n\ \text{prime}} \left( 1-\frac{1}{p} \right) \]
where the product is indexed over the distinct positive prime factors $p$ of $n$.
\end{theorem}
\begin{cproof}
If $n < 0$ then $\varphi(n) = \varphi(-n)$, $|n| = -n$ and $p \mid n$ if and only if $p \mid {-n}$, so the theorem holds for negative integers if and only if it holds for positive integers.

We prove the formula for $n > 0$ by well-founded induction on $\mathbb{Z}^{>0}$ with respect to the relation $R$ defined in \Cref{propPositiveIntegersAsWellFoundedSet}.
\begin{itemize}
\item (\textbf{BC}) $\varphi(1)=1$ and, since no prime $p$ divides $1$, we have $\prod_{p \mid 1\ \text{prime}} \left( 1-\frac{1}{p} \right) = 1$. Hence
\[ 1 \cdot \prod_{p \mid 1\ \text{prime}} \left( 1-\frac{1}{p} \right) = 1 \cdot 1 = 1 \]
as erquired.
\item (\textbf{IS}) Fix $n \ge 1$ and suppose that
\[ \varphi(n) = n \cdot \prod_{p \mid n\ \text{prime}} \left( 1-\frac{1}{p} \right) \]
Let $q > 0$ be prime. We prove that
\[ \varphi(qn) = qn \cdot \prod_{p \mid qn\ \text{prime}} \left( 1-\frac{1}{p} \right) \]
\begin{itemize}
\item Suppose $q \mid n$. Then by we have
\begin{align*}
\varphi(qn) &= q\varphi(n) && \text{by \Cref{exTotientMultiplyByPrime}} \\
&= qn \cdot \prod_{p \mid n\, prime} \left( 1-\frac{1}{p} \right) && \text{by induction hypothesis} \\
&= qn \cdot \prod_{p \mid qn\, prime} \left( 1-\frac{1}{p} \right) && 
\end{align*}
The last equation holds because the fact that $q \mid n$ implies that, for all positive primes $p$, we have $p \mid n$ if and only if $p \mid qn$.
\item Suppose $q \nmid n$. Then $q \perp n$, so we have
\begin{align*}
\varphi(qn) &= \varphi(q)\varphi(n) && \text{by \Cref{thmTotientIsMultiplicative}} \\
&= \varphi(q) \cdot n \cdot \prod_{p \mid n \text{ prime}} \left( 1-\frac{1}{p} \right) && \text{by induction hypothesis} \\
&= (q-1) \cdot n \cdot \prod_{p \mid n \text{ prime}} \left( 1-\frac{1}{q} \right) && \text{by \Cref{exComputationsOfTotients}} \\
&= q\left( 1-\frac{1}{p} \right) n \cdot \prod_{p \mid n \text{ prime}} \left( 1-\frac{1}{p} \right) && \text{rearranging} \\
&= qn \cdot \left(\prod_{p \mid n \text{ prime}} \left( 1 - \frac{1}{p} \right)\right) \cdot \left( 1 - \frac{1}{q} \right) && \text{rearranging} \\
&= qn \cdot \prod_{p \mid qn} \left( 1 - \frac{1}{p} \right) && \text{reindexing the product}
\end{align*}
\end{itemize}
In both cases, we have shown that the formula holds.
\end{itemize}
By induction, we're done.
\end{cproof}

\subsection*{Well-founded relations}

First, we introduce the notion of a \textit{well-founded relation}.

\begin{definition}
\label{defWellFoundedRelation}
\index{relation!well-founded}
\index{relation!ill-founded}
\index{well-founded relation}
\index{ill-founded relation}
Let $X$ be a set. A relation $R$ on $X$ is \textbf{well-founded} if every inhabited subset of $X$ has an \textbf{$R$-minimal} element, in the following sense: for each inhabited $U \subseteq X$, there exists $m \in U$ such that $\neg (x\;R\;m)$ for all $x \in U$. A relation that is not well-founded is called \textbf{ill-founded}.
\end{definition}

\begin{example}
\label{exNIsWellFounded}
The relation $<$ on $\mathbb{N}$ is well-founded---this is just a fancy way of stating the well-ordering principle (\Cref{thmWellOrderingPrinciple}). Indeed, let $U \subseteq \mathbb{N}$ be an inhabited subset. By the well-ordering principle, there exists an element $m \in U$ such that $m \le x$ for all $x \in U$. But this says precisely that $\neg (x < m)$ for all $x \in U$.
\end{example}

\begin{example}
\label{exZNotWellFounded}
However, the relation $<$ on $\mathbb{Z}$ is not well-founded---indeed, $\mathbb{Z}$ is an inhabited subset of $\mathbb{Z}$ with no $<$-least element.
\end{example}

\begin{exercise}
\label{exNIsWellFoundedBySuccessor}
Let ${<}^1$ be the relation on $\mathbb{N}$ defined for $m,n \in \mathbb{N}$ by
\[ m \;{<}^1\; n \quad \Leftrightarrow \quad n = m + 1 \]
Prove that ${<}^1$ is a well-founded relation on $\mathbb{N}$.
\end{exercise}

\begin{proposition}
\label{propWellFoundedIffNoInfiniteDescendingChains}
Let $X$ be a set and let $R$ be a relation on $X$. $R$ is well-founded if and only if there is no infinite $R$-descending chains; that is, there does not exist a sequence $(x_n)_{n \in \mathbb{N}}$ of elements of $X$ such that $x_{n+1}\;R\;x_n$ for all $n \in \mathbb{N}$.
\end{proposition}
\begin{cproof}
We prove the contrapositives of the two directions; that is, $R$ is ill-founded if and only if $R$ has an infinite descending $R$-chain.

\begin{itemize}
\item ($\Rightarrow$) Suppose that $R$ is ill-founded, and let $U \subseteq X$ be an inhabited subset with no $R$-minimal element. Define a sequence $(x_n)_{n \in \mathbb{N}}$ of elements of $X$---in fact, of $U$---recursively as follows:
\begin{itemize}
\item Let $x_0 \in U$ be arbitrarily chosen.
\item Fix $n \in \mathbb{N}$ and suppose $x_0, x_1, \dots, x_n \in U$ have been defined. Since $U$ has no $R$-minimal element, it contains an element which is related to $x_n$ by $R$; define $x_{n+1}$ to be such an element.
\end{itemize}
Then $(x_n)_{n \in \mathbb{N}}$ is an infinite $R$-descending chain
\item ($\Leftarrow$) Suppose there is an infinite $R$-descending chain $(x_n)_{n \in \mathbb{N}}$. Define $U = \{ x_n \mid n \in \mathbb{N} \}$ to be the set of elements in this sequence. Then $U$ has no $R$-minimal element. Indeed, given $m \in U$, we must have $m=x_n$ for some $n \in \mathbb{N}$; but then $x_{n+1} \in U$ and $x_{n+1}\;R\;m$. Hence $R$ is ill-founded.
\end{itemize}
\end{cproof}

\begin{proposition}
\label{propPositiveIntegersAsWellFoundedSet}
Let $\mathbb{Z}^{>0}$ be the set of positive integers and define a relation $R$ on $\mathbb{Z}^{>0}$ by
\[ m\; R\; n \quad \Leftrightarrow \quad n=pm \text{ for some prime } p>0 \]
for all $m,n > 0$. Then $R$ is a well-founded relation on $\mathbb{Z}^{>0}$.
\end{proposition}
\begin{cproof}
Suppose that $(x_n)_{n \in \mathbb{N}}$ is an infinite $R$-descending chain in $\mathbb{Z}^{>0}$. Since $x_{n+1}\;R\;x_n$ for all $n \in \mathbb{N}$, we have $x_n = px_{n+1}$ for some positive prime $p$ for all $n \in \mathbb{N}$. Since all positive primes are greater than or equal to $2$, this implies that $x_n \ge 2x_{n+1}$ for all $n \in \mathbb{N}$.

We prove by strong induction on $n \in \mathbb{N}$ that $x_0 > 2^nx_{n+1}$ for all $n \in \mathbb{N}$.

\begin{itemize}
\item (\textbf{BC}) We proved above that $x_0 \ge 2x_1$. Hence $x_0 > x_1 = 2^0x_1$, as required.
\item (\textbf{IS}) Fix $n \in \mathbb{N}$ and suppose $x_0 > 2^nx_{n+1}$. We want to show $x_0 > 2^{n+1}x_{n+2}$. Well $x_{n+1} > 2x_{n+2}$, as proved above, and hence
\[ x_0 \overset{\text{IH}}{>} 2^nx_{n+1} > 2^n \cdot 2x_{n+2} = 2^{n+1}x_{n+2} \]
as required.
\end{itemize}

By induction, we've shown that $x_0 > 2^nx_{n+1}$ for all $n \in \mathbb{N}$. But $x_{n+1} > 0$ for all $n \in \mathbb{N}$, so $x_0 > 2^n$ for all $n \in \mathbb{N}$. This implies that $x_0$ is greater than every integer, which is a contradiction.

So such a sequence $(x_n)_{n \in \mathbb{N}}$ cannot exist, and by \Cref{propWellFoundedIffNoInfiniteDescendingChains}, the relation $R$ is well-founded.
\end{cproof}

\begin{exercise}
\label{exWellFoundedRelationsAreAsymmetric}
Let $X$ be a set and let $R$ be a well-founded relation on $X$. Given $x,y \in X$, prove that not both $x\; R\; y$ and $y\; R\; x$ are true.
\end{exercise}

\begin{theorem}[Principle of well-founded induction]
\label{thmWellFoundedInduction}
\index{well-founded induction}
\index{induction!on a well-founded relation}
\index{R-induction@$R$-induction}
Let $X$ be a set, let $R$ be a well-founded relation on $X$, and let $p(x)$ be a logical formula concerning elements of $X$. Suppose that for each $x \in X$, the following is true:
\begin{center}
\textit{If $p(y)$ is true for all $R$-predecessors $y$ of $x$, then $p(x)$ is true.}
\end{center}
That is, suppose for each $x \in X$ that
\[ [\forall y \in X,\, (y\; R\; x \Rightarrow p(y))] \Rightarrow p(x) \]
Then $p(x)$ is true for all $x \in X$.
\end{theorem}
\begin{cproof}
Suppose that, for each $x \in X$, if $p(y)$ is true for all $R$-predecessors $y$ of $x$, then $p(x)$ is true. Let
\[ U = \{ x \in X \mid \neg p(x) \} \]
Towards a contradiction, suppose that $p(x)$ is false for some $x \in X$. Then $U$ is inhabited. Since $R$ is well-founded, $U$ has an $R$-minimal element $m \in U$. Now
\begin{enumerate}[(i)]
\item $p(m)$ is false, since $m \in U$.
\item $p(x)$ is true for all $x \in X$ with $x\; R\; m$. To see this, note that if $p(x)$ is false and $x\; R\; m$, then $x \in U$, so that $m\; R\; x$ by $R$-minimality of $m$ in $U$. Since also $x\; R\; m$, this contradicts \Cref{exWellFoundedRelationsAreAsymmetric}.
\end{enumerate}
Since $p(x)$ is true for all $x \in X$ with $x\; R\; m$, by assumption we also have that $p(m)$ is true. But this contradicts our assumption that $m \in U$.

So it must in fact be the case that $U = \varnothing$, so that $p(x)$ is true for all $x \in X$.
\end{cproof}

\begin{exercise}
Prove that the principle of $<$-induction on $\mathbb{N}$ is precisely strong induction. Specifically, prove that the following two statements are equivalent:
\begin{enumerate}[(i)]
\item $p(0)$ is true and, for all $n \in \mathbb{N}$, if $p(k)$ is true for all $k \le n$, then $p(n+1)$ is true;
\item For all $n \in \mathbb{N}$, if $p(k)$ is true for all $k<n$, then $p(n)$ is true.
\end{enumerate}
Strong induction says that we can deduce that $p(n)$ is true for all $n \in \mathbb{N}$ from the knowledge that (i) is true for all $n \in \mathbb{N}$; and $<$-induction tells us that $p(n)$ is true for all $n \in \mathbb{N}$ from the knowledge that (ii) is true for all $n \in \mathbb{N}$. You should prove that (i) and (ii) are equivalent.
\end{exercise}

\begin{example}
\label{exWellFoundedInductionOnNWithSuccessor}
Let ${<}^1$ be the relation on $\mathbb{N}$ defined in \Cref{exNIsWellFoundedBySuccessor}. We prove that the principle of ${<}^1$-induction on $\mathbb{N}$ is precisely strong induction. Specifically, prove that the following two statements are equivalent:
\begin{enumerate}[(i)]
\item $p(0)$ is true and, for all $n \in \mathbb{N}$, if $p(n)$ is true then $p(n+1)$ is true;
\item For all $n \in \mathbb{N}$, if $p(k)$ is true for all $k \in \mathbb{N}$ with $k+1=n$, then $p(n)$ is true.
\end{enumerate}

Weak induction says that we can deduce that $p(n)$ is true for all $n \in \mathbb{N}$ from the knowledge that (i) is true for all $n \in \mathbb{N}$; and ${<}^1$-induction tells us that $p(n)$ is true for all $n \in \mathbb{N}$ from the knowledge that (ii) is true for all $n \in \mathbb{N}$. We prove that (i) and (ii) are equivalent.

\begin{itemize}
\item (i) $\Rightarrow$ (ii). Suppose that $p(0)$ and, for all $n \in \mathbb{N}$, if $p(n)$ is true then $p(n+1)$ is true. We will prove that
\[ [\forall m \in \mathbb{N},\, (n=m+1 \Rightarrow p(m))] \Rightarrow p(n) \]
is true for all $n \in \mathbb{N}$.

So fix $n \in \mathbb{N}$, and assume $\forall m \in \mathbb{N},\, (n=m+1 \Rightarrow p(m))$. We prove $p(n)$ is true.
\begin{itemize}
\item If $n=0$ then we're done, since $p(0)$ is true by assumption.
\item If $n>0$ then $n=m+1$ for some $m \in \mathbb{N}$. By our assumption, we have $\forall m \in \mathbb{N},\, (n=m+1 \Rightarrow p(m))$, and so in particular, $p(m)$ is true. By the weak induction step, we have $p(m) \Rightarrow p(m+1)$ is true. But then $p(m+1)$ is true. Since $n=m+1$, we have that $p(n)$ is true.
\end{itemize}
In any case, we've proved that $p(n)$ is true, as required.

\item (ii) $\Rightarrow$ (i). For $n \in \mathbb{N}$, denote the following statement by $H(n)$
\[ [\forall m \in \mathbb{N},\, (n=m+1 \Rightarrow p(m))] \Rightarrow p(n) \]
Assume $H(n)$ is true for all $n \in \mathbb{N}$. We prove that $p(0)$ is true and, for all $n \in \mathbb{N}$, if $p(n)$ is true then $p(n+1)$ is true.
\begin{itemize}
\item $p(0)$ is true. Indeed, for any $m \in \mathbb{N}$ we have that $0=m+1$ is false, so the statement ~$0=m+1 \Rightarrow p(m)$ is true. Hence $\forall m \in \mathbb{N},\, (0=m+1 \Rightarrow p(m))$ is true. Since $H(0)$ is true, it follows that $p(0)$ is true.
\item Fix $n \in \mathbb{N}$ and suppose $p(n)$ is true. By $H(n+1)$, we have that if $p(n+1)$ is true for all $m \in \mathbb{N}$ with $m+1=n+1$, then $p(n+1)$ is true. But the only $m \in \mathbb{N}$ such that $m+1=n+1$ is $n$ itself, and $p(n)$ is true by assumption; so by $H(n+1)$, we have $p(n+1)$, as required.
\end{itemize}
\end{itemize}

Hence the two induction principles are equivalent.
\end{example}

\begin{example}

\end{example}

\subsection*{Structural induction from well-founded induction}

We will now derive the principle of structural induction in terms of the principle of well-founded induction. To do this, we need to associate to each inductively defined set $X$ a corresponding well-founded relation $R_X$, such that well-founded induction on $R_X$ corresponds with structural induction on $X$.

\begin{definition}
\label{defRX}
Let $X$ be an inductively defined set. Define a relation $R_X$\nindex{RX}{$R_X$}{relation associated with an inductively defined set} on $X$ as follows: for all $x,y \in X$, $x\;R_X\;y$ if and only if
\[ y=f(x_1,x_2,\dots,x_n) \]
for some constructor $f$ of arity $n$ and elements $x_1,x_2,\dots,x_n$, such that $x_i=x$ for some $i \in [n]$.
\end{definition}

\begin{example}
Let $\mathbb{N}$ be the set of natural numbers, taken to be inductively defined in the usual way. Since the only constructor is the successor operation, we must have for $m,n \in \mathbb{N}$ that
\[ m\; R_{\mathbb{N}}\; n \quad \Leftrightarrow \quad n=m+1 \]
This is precisely the relation ${<}^1$ from \Cref{exNIsWellFoundedBySuccessor}. We already established that structural induction on $\mathbb{N}$ is precisely weak induction (\Cref{exStructuralInductoinOnNIsWeakInduction}), and that well-founded induction on ${<}^1$ is also precisely weak induction (\Cref{exWellFoundedInductionOnNWithSuccessor}).
\end{example}

\begin{example}
Let $P$ be a set of propositional variables and let $L(P)$ be the set of propositional formulae built from variables in $P$ and the logical operators $\wedge$, $\vee$, $\Rightarrow$ and $\neg$.

Then $R=R_{L(P)}$ is the relation defined for $s,t \in L(P)$ by letting $s \;R\; t$ if and only if
\[ t \in \{ s \wedge u,\ u \wedge s,\ s \vee u,\ u \vee s,\ s \Rightarrow u,\ u \Rightarrow s,\ \neg s \} \]
for some $u \in L(P)$.
\end{example}

The plan for the rest of this section is to demonstrate that structural induction follows from well-founded induction. To do this, we prove that the relation $R_X$ associated with an inductively defined set $X$ is well-founded, and then we prove that structural induction on $X$ is equivalent to well-founded induction on $R_X$.

To simplify our proofs, we introduce the notion of \textit{rank}. The rank of an element $x$ of an inductively defined set $X$ is a natural number which says how many constructors need to be applied in order to obtain $x$.

% \begin{definition}
% \label{defRank}
% \index{rank}
% Let $X$ be an inductively defined set. The function $\mathrm{rank} : X \to \mathbb{N}$ is defined recursively as follows:
% \begin{itemize}
% \item If $b$ is a basic element of $X$, then $\mathrm{rank}(b)=0$.
% \item Let $f$ be a constructor of arity $n$ and let $x_1,x_2,\dots,x_n \in X$. Then
% \[ \mathrm{rank}(f(x_1,x_2,\dots,x_n)) = \mathrm{max} \{ \mathrm{rank}(x_1), \mathrm{rank}(x_2), \dots, \mathrm{rank}(x_n) \} + 1 \]
% \end{itemize}
% \end{definition}

Note that $\mathrm{rank} : X \to \mathbb{N}$ is a well-defined function, since by the conditions listed in \Cref{defInductivelyDefinedSet}, every element of $X$ is either basic or has a unique representation in the form $f(x_1,x_2,\dots,x_n)$ for some constructor $f$ and elements $x_1,x_2,\dots,x_n \in X$.

\begin{example}
\label{exRankOnN}
The rank function on the inductively defined set of natural numbers is fairly boring. Indeed, it tells us that
\begin{itemize}
\item $\mathrm{rank}(0)=0$; and
\item $\mathrm{rank}(n+1) = \mathrm{rank}(n) + 1$ for all $n \in \mathbb{N}$.
\end{itemize}
It can easily be seen that $\mathrm{rank}(n)=n$ for all $n \in \mathbb{N}$. This makes sense, since $n$ can be obtained from $0$ by iterating the successor operation $n$ times.
\end{example}

\begin{lemma}
Let $X$ be an inductively defined set. The relation $R_X$ defined in \Cref{defRX} is well-founded.
\end{lemma}
\begin{cproof}

\end{cproof}

\begin{cproof}[of \Cref{thmStructuralInduction}]
\label{prfStructuralInduction}
\todo{Write proof}
\end{cproof}

\todo{Examples and exercises}

\index{induction!structural|)}