\section{Transfinite induction}
\label{secTransfiniteInduction}

\todo{}

\begin{axiom}[Axiom of choice]
\label{axAxiomOfChoice}
\index{axiom of choice}
For any family of inhabited sets $\{ X_i \mid i \in I \}$, there is a function $f : I \to \bigcup_{i \in I} X_i$ such that $f(i) \in X_i$ for each $i \in I$. The function $f$ is called a \textbf{choice function} for $\{ X_i \mid i \in I \}$.
\end{axiom}

\todo{}

\subsection*{Well-ordered sets}

\begin{definition}
\label{defWellOrder}
Let $X$ be a set. A \textbf{well-order} on $X$ is a well-founded total order relation $\preceq$.
\end{definition}

\begin{theoremac}[Well-ordering principle]
Every set can be well-ordered.
\end{theoremac}

\begin{cproof}
\todo{}
\end{cproof}

\todo{}

\subsection*{Ordinal numbers}

\todo{}

\begin{axiom}[Axiom of foundation]
\label{axFoundation}
Every set   
\end{axiom}

\todo{}

\begin{definition}
\label{defVonNeumannOrdinal}
A \textbf{von Neumann ordinal} is a set $X$ such that:
\begin{enumerate}[(a)]
\item $X$ is \textbf{pure}---that is, every element of $X$ is itself a set
\item $X$ is \textbf{transitive}---that is, every element of $X$ is itself a set and $A \subseteq X$ 
\end{enumerate}
\end{definition}

\todo{}

\subsection*{Ordinal arithmetic}

\todo{}

\subsection*{Cardinal numbers as ordinal numbers}

\todo{}

\begin{definition}

\end{definition}