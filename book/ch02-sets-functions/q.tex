\subsection*{Set notation}

\begin{chapex}
Express the following sets in the indicated form of notation.
\begin{enumerate}[(a)]
\item $\{ n \in \mathbb{Z} \mid n^2 < 20 \}$ in list notation;
\item $\{ 4k+3 \mid k \in \mathbb{N} \}$ in implied list notation;
\item The set of all odd multiples of six in set-builder notation;
\item The set $\{ 1, 2, 5, 10, 17, \dots, n^2+1, \dots \}$ in set-builder notation.
\end{enumerate}
\end{chapex}

\subsection*{Set operations}

\begin{chapex}
\label{cqPowerSetRespectSetOperations}
For each of the following statements, determine whether it is true for all sets $X,Y$, false for all sets $X,Y$, or true for some choices of $X$ and $Y$ and false for others.
\begin{multicols}{2}
\begin{enumerate}[(a)]
\item $\mathcal{P}(X \cup Y) = \mathcal{P}(X) \cup \mathcal{P}(Y)$
\item $\mathcal{P}(X \cap Y) = \mathcal{P}(X) \cap \mathcal{P}(Y)$
\item $\mathcal{P}(X \times Y) = \mathcal{P}(X) \times \mathcal{P}(Y)$
\item $\mathcal{P}(X \setminus Y) = \mathcal{P}(X) \setminus \mathcal{P}(Y)$
\end{enumerate}
\end{multicols}
\end{chapex}

\Crefrange{cqSymmetricDifferenceBegin}{cqSymmetricDifferenceEnd} concern the \textit{symmetric difference} of sets, defined below.

\begin{definition}
\label{defSymmetricDifference}
\index{symmetric difference}
\nindex{symmdiff}{$\triangle$}{symmetric difference}
The \textbf{symmetric difference} of sets $X$ and $Y$ is the set $X \symmdiff Y$ \inlatex{triangle}\lindexmmc{triangle}{$\triangle$} defined by
\[ X \symmdiff Y = \{ a \mid a \in X \text{ or } a \in Y \text{ but } a \not\in X \cap Y \} \]
\end{definition}

\begin{chapex}
\label{cqSymmetricDifferenceBegin}
Prove that $X \symmdiff Y = (X \setminus Y) \cup (Y \setminus X) = (X \cup Y) \setminus (X \cap Y)$ for all sets $X$ and $Y$.
\end{chapex}

\begin{chapex}
Let $X$ be a set. Prove that $X \symmdiff X = \varnothing$ and $X \symmdiff \varnothing = X$.
\end{chapex}

\begin{chapex}
Let $X$ and $Y$ be sets. Prove that $X = Y$ if and only if $X \symmdiff Y = \varnothing$.
\end{chapex}

\begin{chapex}
Prove that sets $X$ and $Y$ are disjoint if and only if $X \symmdiff Y = X \cup Y$.
\end{chapex}

\begin{chapex}
Prove that $X \symmdiff (Y \symmdiff Z) = (X \symmdiff Y) \symmdiff Z$ for all sets $X$, $Y$ and $Z$.
\hintlabel{cqSymmetricDifferenceEnd}{%
The temptation is to write a long string of equations, but it is far less painful to prove this by double containment, splitting into cases where needed. An even less painful approach is to make a cunning use of truth tables.
}
\end{chapex}

\begin{definition}
\label{defOpenSubsetOfR}
\index{open!subset of $\mathbb{R}$}
A subset $U \subseteq \mathbb{R}$ is \textbf{open} if, for all $a \in U$, there exists $\delta > 0$ such that $(a-\delta, a+\delta) \subseteq U$.
\end{definition}

In Questions \Crefrange{cqOpenSubsetsOfRBegin}{cqOpenSubsetsOfREnd} you will prove some elementary facts about open subsets of $\mathbb{R}$.

\begin{chapex}
\label{cqOpenSubsetsOfRBegin}
For each of the following subsets of $\mathbb{R}$, determine (with proof) whether it is open:
\begin{multicols}{3}
\begin{enumerate}[(a)]
\item $\varnothing$;
\item $(0,1)$;
\item $(0,1]$;
\item $\mathbb{Z}$;
\item $\mathbb{R} \setminus \mathbb{Z}$;
\item $\mathbb{Q}$.
\end{enumerate}
\end{multicols}
\end{chapex}

\begin{chapex}
Prove that a subset $U \subseteq \mathbb{R}$ is open if and only if, for all $a \in U$, there exist $u,v \in \mathbb{R}$ such that $u<a<v$ and $(u,v) \subseteq U$.
\end{chapex}

\begin{chapex}
In this question you will prove that the intersection of finitely many open sets is open, but the intersection of infinitely many open sets might not be open.
\begin{enumerate}[(a)]
\item Let $n \ge 1$ and suppose $U_1, U_2, \dots, U_n$ are open subsets of $\mathbb{R}$. Prove that the intersection $U_1 \cap U_2 \cap \cdots \cap U_n$ is open.
\item Prove that $(0,1+\frac{1}{n})$ is open for all $n \ge 1$, but that $\bigcap_{n \ge 1} (0,1+\textstyle\frac{1}{n})$ is not open.
\end{enumerate}
\end{chapex}

\begin{chapex}
\label{cqOpenSubsetsOfREnd}
Prove that a subset $U \subseteq \mathbb{R}$ is open if and only if it can be expressed as a union of open intervals---more precisely, $U \subseteq \mathbb{R}$ is open if and only if, for some indexing set $I$, there exist real numbers $a_i, b_i$ for each $i \in I$, such that $U = \bigcup_{i \in I} (a_i, b_i)$.
\end{chapex}

\subsection*{Functions}

\begin{chapex}
Show that there is only one function whose codomain is empty. What is its domain?
\hintlabel{cqFunctionEmptyCodomain}{%
Given a function $f : X \to \varnothing$, we must have $f(a) \in \varnothing$ for each $a \in X$.
}
\end{chapex}

\begin{definition}
\label{defEvenOddFunction}
\index{even!function}
\index{odd!function}
A function $f : \mathbb{R} \to \mathbb{R}$ is \textbf{even} if $f(-x)=f(x)$ for all $x \in \mathbb{R}$, and it is \textbf{odd} if $f(-x)=-f(x)$ for all $x \in \mathbb{R}$.
\end{definition}

\begin{chapex}
Let $n \in \mathbb{N}$. Prove that the function $f : \mathbb{R} \to \mathbb{R}$ defined by $f(x)=x^n$ for all $x \in \mathbb{R}$ is even if and only if $n$ is even, and odd if and only if $n$ is odd.
\end{chapex}

\begin{chapex}
Prove that there is a unique function $f : \mathbb{R} \to \mathbb{R}$ that is both even and odd.
\end{chapex}

\begin{chapex}
Let $U \subseteq \mathbb{R}$, and let $\chi_U : \mathbb{R} \to \{0,1\}$ be the indicator function of $U$.
\begin{enumerate}[(a)]
\item Prove that $\chi_U$ is an even function if and only if $U = \{ -u \mid u \in U \}$;
\item Prove that $\chi_U$ is an odd function if and only if $\mathbb{R} \setminus U = \{ -u \mid u \in U \}$.
\end{enumerate}
\end{chapex}

\begin{chapex}
Prove that for every function $f : \mathbb{R} \to \mathbb{R}$, there is a unique even function $g : \mathbb{R} \to \mathbb{R}$ and a unique odd function $h : \mathbb{R} \to \mathbb{R}$ such that $f(x)=g(x)+h(x)$ for all $x \in \mathbb{R}$.
\hintlabel{cqEveryFunctionIsUniquelySumOfEvenAndOdd}{%
Consider $f(x)+f(-x)$ and $f(x)-f(-x)$ for $x \in \mathbb{R}$.
}
\end{chapex}

\begin{chapex}
Let $\{ \theta_n : [n] \to [n] \mid n \in \mathbb{N} \}$ be a family of functions such that $f \circ \theta_m = \theta_n \circ f$ for all $f : [m] \to [n]$. Prove that $\theta_n = \mathrm{id}_{[n]}$ for all $n \in \mathbb{N}$.
\hintlabel{cqNaturalTransformationFromIdToId}{%
Fix $n \in \mathbb{N}$ and consider how $\theta_n$ interacts with functions $f : [1] \to [n]$.
}
\end{chapex}

\begin{chapex}
Let $X$ be a set and let $U, V \subseteq X$. Describe the indicator function $\chi_{U \symmdiff V}$ of the symmetric difference of $U$ and $V$ (see \Cref{defSymmetricDifference}) in terms of $\chi_U$ and $\chi_V$.
\end{chapex}

\subsection*{Images and preimages}

In \Crefrange{cqComputeImageBegin}{cqComputeImageEnd}, find the image $f[U]$ of the subset $U$ of the codomain of the function $f$ described in the question.

\begin{chapex}
\label{cqComputeImageBegin}
$f : \mathbb{R} \to \mathbb{R}$; $f(x) = \sqrt{1+x^2}$ for all $x \in \mathbb{R}$; $U = \mathbb{R}$.
\end{chapex}

\begin{chapex}
$f : \mathbb{Z} \times \mathbb{Z} \to \mathbb{Z}$; $f(a,b) = a+2b$ for all $(a,b) \in \mathbb{Z} \times \mathbb{Z}$; $U = \{ 1 \} \times \mathbb{Z}$.
\end{chapex}

\begin{chapex}
$f : \mathbb{N} \to \mathcal{P}(\mathbb{N})$; $f(0) = \varnothing$ and $f(n+1) = f(n) \cup \{ n \}$ for all $n \in \mathbb{N}$; $U = \mathbb{N}$.
\end{chapex}

\begin{chapex}
$f : \mathbb{R}^{\mathbb{R}} \to \mathbb{R}^{\mathbb{R}}$ (where $\mathbb{R}^{\mathbb{R}}$ is the set of all functions $\mathbb{R} \to \mathbb{R}$); $f(h)(x) = h(|x|)$ for all $h \in \mathbb{R}^{\mathbb{R}}$ and all $x \in \mathbb{R}$; $U = \mathbb{R}^{\mathbb{R}}$.
\hintlabel{cqComputeImageEnd}{%
Begin by observing that each $h \in f[\mathbb{R}^{\mathbb{R}}]$ is an even function, in the sense of \Cref{defEvenOddFunction}.
}
\end{chapex}

In \Crefrange{cqComputePreimageBegin}{cqComputePreimageEnd}, find the preimage $f^{-1}[V]$ of the subset $V$ of the codomain of the function $f$ described in the question.

\begin{chapex}
\label{cqComputePreimageBegin}
$f : \mathbb{R} \to \mathbb{R}$; $f(x) = \sqrt{1+x^2}$ for all $x \in \mathbb{R}$; $V = (-5,5]$.
\end{chapex}

\begin{chapex}
$f : \mathbb{Z} \times \mathbb{Z} \to \mathbb{Z}$; $f(a,b) = a+2b$ for all $(a,b) \in \mathbb{Z} \times \mathbb{Z}$; $V = \{ n \in \mathbb{Z} \mid n \text{ is odd} \}$.
\end{chapex}

\begin{chapex}
\label{cqComputePreimageEnd}
$f : \mathcal{P}(\mathbb{N}) \times \mathcal{P}(\mathbb{N}) \to \mathcal{P}(\mathbb{N})$; $f(A,B) = A \cap B$ for all $(A,B) \in \mathcal{P}(\mathbb{N}) \times \mathcal{P}(\mathbb{N})$; $V = \{ \varnothing \}$.
\end{chapex}

\begin{chapex}
Let $f : X \to Y$ be a function. For each of the following statements, either prove it is true or find a counterexample.
\begin{multicols}{2}
\begin{enumerate}[(a)]
\item $U \subseteq f^{-1}[f[U]]$ for all $U \subseteq X$;
\item $f^{-1}[f[U]] \subseteq U$ for all $U \subseteq X$;
\item $V \subseteq f[f^{-1}[V]]$ for all $V \subseteq Y$;
\item $f[f^{-1}[V]] \subseteq V$ for all $V \subseteq Y$.
\end{enumerate}
\end{multicols}
\end{chapex}

\begin{chapex}
Let $f : X \to Y$ be a function, let $A$ be a set, and let $p : X \to A$ and $i : A \to Y$ be functions such that the following conditions hold:
\begin{enumerate}[(i)]
\item $i$ is injective;
\item $i \circ p = f$; and
\item If $q : X \to B$ and $j : B \to Y$ are functions such that $j$ is injective and $j \circ q = f$, then there is a unique function $u : A \to B$ such that $j \circ u = i$.
\end{enumerate}
Prove that there is a unique bijection $v : A \to f[X]$ such that $i(a)=v(a)$ for all $a \in f[X]$.
\hintlabel{cqUniversalPropertyOfImage}{%
This problem is very fiddly. First prove that conditions (i)--(iii) are satisfied when $A = f[X]$ and $p$ and $i$ are chosen appropriately. Then condition (iii) in each case (for $A$ and for $f[X]$) defines functions $v : A \to f[X]$ and $w : f[X] \to A$, and gives uniqueness of $v$. You can prove that these functions are mutually inverse using the `uniqueness' part of condition (iii).
}
\end{chapex}

\begin{chapex}
Let $f : X \to Y$ be a function and let $U, V \subseteq Y$. Prove that:
\begin{enumerate}[(a)]
\item $f^{-1}[U \cap V] = f^{-1}[U] \cap f^{-1}[V]$;
\item $f^{-1}[U \cup V] = f^{-1}[U] \cup f^{-1}[V]$; and
\item $f^{-1}[Y \setminus U] = X \setminus f^{-1}[U]$.
\end{enumerate}
Thus preimages preserve the basic set operations.
\end{chapex}

\begin{chapex}
Let $f : X \to Y$ and $g : Y \to Z$ be functions.
\begin{enumerate}[(a)]
\item Prove that $(g \circ f)[U] = g[f[U]]$ for all $U \subseteq X$;
\item Prove that $(g \circ f)^{-1}[W] = f^{-1}[g^{-1}[W]]$ for all $W \subseteq Z$.
\end{enumerate}
\end{chapex}

\subsection*{Injections, surjections and bijections}

\begin{chapex}
\begin{enumerate}[(a)]
\item Prove that, for all functions $f : X \to Y$ and $g : Y \to Z$, if $g \circ f$ is bijective, then $f$ is injective and $g$ is surjective.
\item Find an example of a function $f : X \to Y$ and a function $g : Y \to Z$ such that $g \circ f$ is bijective, $f$ is not surjective and $g$ is not injective.
\end{enumerate}
\hintlabel{cqCompositeOfFunctionsIsBijection}{%
Avoid the temptation to prove either part of this question by contradiction. For (a), a short proof is available directly from the definitions of `injection' and `surjection'. For (b), find as simple a counterexample as you can.
}
\end{chapex}

\begin{chapex}
\label{cqInjectionSurjectionBijection}
For each of the following pairs $(U,V)$ of subsets of $\mathbb{R}$, determine whether the specification `$f(x) = x^2-4x+7$ for all $x \in U$' defines a function $f : U \to V$ and, if it does, determine whether $f$ is injective and whether $f$ is surjective.
\begin{multicols}{2}
\begin{enumerate}[(a)]
\item $U = \mathbb{R}$ and $V = \mathbb{R}$;
\item $U = (1, 4)$ and $V = [3, 7)$;
\item $U = [3, 4)$ and $V = [4, 7)$;
\item $U = (3, 4]$ and $V = [4, 7)$;
\item $U = [2, \infty)$ and $V = [3, \infty)$;
\item $U = [2,\infty)$ and $V = \mathbb{R}$.
\end{enumerate}
\end{multicols}
\end{chapex}

\begin{chapex}
For each of the following pairs of sets $X$ and $Y$, find (with proof) a bijection $f : X \to Y$.
\begin{enumerate}[(a)]
\item $X = \mathbb{Z}$ and $Y = \mathbb{N}$;
\item $X = \mathbb{R}$ and $Y = (-1,1)$;
\item $X = [0,1]$ and $Y = (0,1)$;
\item $X = [a,b]$ and $Y = (c,d)$, where $a,b,c,d \in \mathbb{R}$ with $a<b$ and $c<d$.
\end{enumerate}
\end{chapex}

\begin{chapex}
Prove that the function $f : \mathbb{N} \times \mathbb{N} \to \mathbb{N}$ defined by $f(a,b) = \dbinom{a+b+1}{2} + b$ for all $(a,b) \in \mathbb{N} \times \mathbb{N}$ is a bijection.
\hintlabel{cqElementaryBijectionNTimesNToN}{%
Start by proving that $\dbinom{m}{2} < \dbinom{m+1}{2}$ for all $m \ge 1$. Deduce that, for all $n \in \mathbb{N}$, there is a unique natural number $k$ such that $\dbinom{k+1}{2} \le n < \dbinom{k+2}{2}$. Can you see what this has to do with the function $f$?
}
\end{chapex}