Now that we have a precise way of reasoning mathematically, it's time to start doing some mathematics!

In \Cref{chGettingStarted} we gave a preliminary definition of a `set' as a collection of objects (\Cref{defSetsPreliminary}), but then we focused almost exclusively on the number sets $\mathbb{N}$, $\mathbb{Z}$, $\mathbb{Q}$, $\mathbb{R}$ and $\mathbb{C}$ in \Cref{chGettingStarted} and \Cref{chLogicalStructure}.

Our first task in this chapter, in \Cref{secSets}, is to make the notion of a set slightly more precise, and to get comfortable with reasoning about sets in the abstract---this is extremely important, as sets are the building blocks of pure mathematics. We will study how different sets relate to one another, and how to build new sets out of old.

Just as fundamental as sets, the concept of a \textit{function} is central to almost every mathematical field. We will use functions heavily throughout the book---they are so important that we have devoted not one, but \textit{two} sections to them. Our first exposure to functions in in \Cref{secFunctions}, where we will define the notion of a function and explore their basic properties.

In \Cref{secInjectionsSurjections} we study two properties that functions might have: \textit{injectivity} and \textit{surjectivity}. These conditions are used to compare sizes of sets, amongst other things, and they arise frequently in areas of mathematics where functions are used.

\todo{Introduce relations}