% !TeX root = ../../book.tex

%
% Contents and indices
%

% Contents
\setcounter{tocdepth}{1}

% Indices
\indexsetup{%
    othercode={
        \fancyhead[RE]{{\color{hdcol}\fontfamily{bch}\itshape \indexname}}
        \fancyhead[LO]{{\color{hdcol}\fontfamily{bch}\itshape \indexname}}
        \fancyhead[RO,LE]{\thepage}\fancyfoot[C]{\thepage}
    }
}
\makeindex[title={Index of topics}]
\makeindex[name=notation, title={Index of notation}]
\makeindex[name=vocabulary, title={Index of vocabulary}]
\makeindex[name=latex, columns=1, title={Index of \LaTeX{} commands}]



%
% Transitions
%

% Counter to make updating filenames easier
\newcommand{\currentchapter}{}

% Commands to be executed at the beginning of each section
\newcommand{\secbegin}[1]{%
    \label[section]{#1}
    \hintsection{\Cref*{#1}}
}

% Commands to be executed at the beginning/end of each chapter exercises section
\newcommand{\chexbegin}[1]{%
    \newpage%
    \hintsection{\Cref*{#1} exercises}
    \setcounter{section}{4}%
    \setcounter{chapex}{0}%
    \renewcommand{\thesection}{\thechapter.E}%
    \section{Chapter \thechapter{} exercises}%
    %\chexwarning
}

\newcommand{\chexend}{%
    \renewcommand{\thesection}{\thechapter.\arabic{section}}%
}

% End-of-section summaries
\newenvironment{tldr}[1]{%
\newpage
\subsection*{TL;DR --- summary of \Cref{#1}}
}{%
}

\newenvironment{tldrlist}{%
\begin{itemize}[labelwidth=50pt,labelsep=30pt,align=parleft]
}{%
\end{itemize}
}

\newcommand{\tldrtag}[1]{\ref{#1}}

\newcommand{\tldritem}[1]{\item[\tldrtag{#1}]}