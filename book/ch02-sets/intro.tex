We saw an informal definition of a \textit{set} in \Cref{chGettingStarted}, but so far the only sets that we have seen are the number sets ($\mathbb{N}$, $\mathbb{Z}$ and so on).

In \Cref{secSets}, we will study sets in the abstract---in particular, the sets that we study are arbitrary collections of mathematical objects, not just numbers. This is essential for further study in pure mathematics, since most (if not all) areas of pure mathematics concern certain kinds of sets!

In \Cref{secSetOperations}, we will then define some operations that allow us to form new sets out of old sets, and prove some identities of an algebraic nature that are closely related to the rules governing logical operators and quantifiers that we saw in \Cref{chLogicalStructure}.

In \Cref{secFunctions} we introduce the notion of a \textit{function}. It is likely (but not assumed) that you have seen functions before, such as real-valued functions in calculus or linear transformations in linear algebra. However, we will study functions in the abstract; the `inputs' and `outputs' of our functions need not be numbers, vectors or points in space; they can be anything at all---in fact, the inputs or outputs to our functions might themselves be functions!

The rest of \Cref{secFunctions} is devoted to proving basic properties about functions. We zoom in on two properties in particular in \Cref{secInjectionsSurjections}, namely \textit{injectivity} and \textit{surjectivity}. These properties allow us to compare the sizes of sets (even infinite ones!); we will use them extensively in \Cref{chCombinatorics,chInfinity} for doing exactly that.