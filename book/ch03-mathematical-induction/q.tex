\subsection*{Recursive definitions}

In \Crefrange{cqRecursiveDefinitionsOfArithmeticOperationsBegin}{cqRecursiveDefinitionsOfArithmeticOperationsEnd}, use the recursive definitions of addition, multiplication and exponentiation directly to prove the desired equation.

\begin{chapex}
\label{cqRecursiveDefinitionsOfArithmeticOperationsBegin}
$1+3=4$
\end{chapex}

\begin{chapex}
$0+5=5$
\end{chapex}

\begin{chapex}
$2 \cdot 3 = 6$
\end{chapex}

\begin{chapex}
$0 \cdot 5 = 0$
\end{chapex}

\begin{chapex}
\label{cqRecursiveDefinitionsOfArithmeticOperationsEnd}
$2^3=8$
\end{chapex}

\begin{chapex}
Give a recursive definition of new quantifiers $\exists^{=n}$ for $n \in \mathbb{N}$, where given a set $X$ and a predicate $p(x)$, the logical formula $\exists^{=n} x \in X,~ p(x)$ means `there are exactly $n$ elements $x \in X$ such that $p(x)$ is true'. That is, define $\exists^{=0}$, and then define $\exists^{=n+1}$ in terms of $\exists^{=n}$.
\hintlabel{cqRecursiveDefinitionOfExistsExactlyNQuantifier}{%
For example, if there are exactly $3$ elements of $X$ making $p(x)$ true, then that means that there is some $a \in X$ such that $p(a)$ is true, and there are exactly two elements $x \in X$ \textit{other than} $a$ making $p(x)$ true.
}
\end{chapex}

\begin{chapex}
Use the recursive definition of binomial coefficients (\Cref{defBinomialCoefficientRecursive}) to prove directly that $\dbinom{4}{2} = 6$.
\end{chapex}

\begin{chapex}
\begin{enumerate}[(a)]
\item Find the number of trailing $0$s in the decimal expansion of $41!$.
\item Find the number of trailing $0$s in the binary expansion of $41!$.
\end{enumerate}
\hintlabel{cqTrailingZerosInFactorial}{%
The number of trailing zeros in the base-$b$ expansion of a natural number $n$ is the greatest natural number $r$ such that $b^r$ divides $n$. How many times does $10$ go into $41!$? How many times does $2$ go into $41!$?
}
\end{chapex}

\begin{chapex}
Let $N$ be a set, let $z \in N$ and let $s : N \to N$. Prove that $(N, z, s)$ is a notion of natural numbers (in the sense of \Cref{defNotionOfNaturalNumbers}) if and only if, for every set $X$, every element $a \in X$ and every function $f : X \to X$, there is a unique function $h : N \to X$ such that $h(z)=a$ and $h \circ f = s \circ h$.
\end{chapex}

\subsection*{Proofs by induction}

\begin{chapex}
Let $a \in \mathbb{N}$ and assume that the last digit in the decimal expansion of $a$ is $6$. Prove that the last digit in the decimal expansion of $a^n$ is $6$ for all $n \ge 1$.
\end{chapex}

\begin{chapex}
Let $f : \mathbb{R} \to \mathbb{R}$ be a function such that $f(0) > 0$ and $f(x+y)=f(x)f(y)$ for all $x,y \in \mathbb{R}$. Prove that there is some positive real number $a$ such that $f(x)=a^x$ for all \textit{rational} numbers $x$..
\hintlabel{cqHomomorphismFromRPlusToRTimes}{%
Start by proving by induction on $n \in \mathbb{N}$ that $f(n) = a^n$ for all $n \in \mathbb{N}$. Then deduce that $f(n) = a^n$ for all $n \in \mathbb{Z}$, and finally deduce that $f(x)=a^x$ for all $x \in \mathbb{Q}$.
}
\end{chapex}