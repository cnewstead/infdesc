\section{Seven deadly sins}
\label{secSevenDeadlySins}

Now that we have seen examples of features that constitute an \textit{effective} proof, we conclude this section by focusing on some features that make a proof less effective or even incorrect. We will refer to these features as \textit{deadly sins}.

\begin{deadlysin}[Abuse of variables]
\label{dsAbuseOfVariables}
Using variables without saying what they are.
\end{deadlysin}

\todo{Examples}

\begin{deadlysin}[Contradiction sandwich]
\label{dsContradictionSandwich}
Proving a proposition $p$ by assuming $p$ is false, proving $p$ directly, and then saying a contradiction has been reached to the assumption that $p$ is false.
\end{deadlysin}

\todo{Examples}

\begin{deadlysin}[Proof salad]
\label{dsWordSalad}
Writing a jumble of words or notation that do not form readable phrases or sentences.
\end{deadlysin}

\todo{Examples}

\begin{deadlysin}[Proof by definition]
\label{dsProofByDefinition}
Justifying a claim `by definition' when the claim does not follow immediately from a definition.
\end{deadlysin}

\todo{Examples}

\begin{deadlysin}[Proof by condescension]
\label{dsProofByCondescension}
Deducing that a result is true by saying it is clear or obvious.
\end{deadlysin}

\todo{Examples}

\begin{deadlysin}[Proof by intuition]
\label{dsProofByIntuition}
Giving an intuitive reason for why a result is true without formal justification.
\end{deadlysin}

\todo{Examples}

\begin{deadlysin}[Proof by backwards logic---\textit{snenop sudom}]
\label{dsBackwardsLogic}
Deducing that a proposition $p$ is true by deriving a true conclusion from $p$.
\end{deadlysin}

\todo{Examples}