% !TeX root = ../../book.tex
\begin{chapex}
The video-sharing website \textit{YouTube} assigns to each video a unique identifier, which is a string of 11 characters from the set
\[ \{ \mathtt{A}, \mathtt{B}, \dots, \mathtt{Z}, \mathtt{a}, \mathtt{b}, \dots, \mathtt{z}, \mathtt{0}, \mathtt{1}, \mathtt{2}, \mathtt{3}, \mathtt{4}, \mathtt{5}, \mathtt{6}, \mathtt{7}, \mathtt{8}, \mathtt{9}, \text{\texttt{-}}, \text{\texttt{\_}} \} \]
This string is actually a natural number expressed in base-64, where the characters in the above set represent the numbers $0$ through $63$, in the same order---thus $\mathtt{C}$ represents $2$, $\mathtt{c}$ represents $28$, $\mathtt{3}$ represents $55$, and \texttt{\_} represents $63$. According to this schema, find the natural number whose base-64 expansion is $\mathtt{dQw4w9WgXcQ}$, and find the base-64 expansion of the natural number $7\,159\,047\,702\,620\,056\,984$.
\end{chapex}

\begin{chapex}
Let $a, b, c, d \in \mathbb{Z}$. Under what conditions is $(a+b\sqrt{2})(c+d\sqrt{2})$ an integer?
\end{chapex}

\begin{chapex}
Suppose an integer $m$ leaves a remainder of $i$ when divided by $3$, and an integer $m$ leaves a remainder of $j$ when divided by $3$, where $0 \le i,j < 3$. Prove that $m+n$ and $i+j$ leave the same remainder when divided by $3$.
\end{chapex}

\begin{chapex}
What are the possible integers of $n^2$ when divided by $3$, where $n \in \mathbb{Z}$?
\end{chapex}

\begin{definition}
A set $X$ is \textbf{closed} under an operation $\odot$ if, whenever $a$ and $b$ are elements of $X$, $a \odot b$ is also an element of $X$.
\end{definition}

In \Crefrange{cqClosureOfNumberSetsBegin}{cqClosureOfNumberSetsEnd}, determine which of the number sets $\mathbb{N}$, $\mathbb{Z}$, $\mathbb{Q}$ and $\mathbb{R}$ are closed under the operation $\odot$ defined in the question.

\begin{multicols}{2}
\begin{chapex}
\label{cqClosureOfNumberSetsBegin}
$a \odot b = a + b$
\end{chapex}

\begin{chapex}
$a \odot b = a - b$
\end{chapex}

\begin{chapex}
$a \odot b = (a-b)(a+b)$
\end{chapex}

\begin{chapex}
$a \odot b = (a-1)(b-1) + 2(a+b)$
\end{chapex}

\begin{chapex}
$a \odot b = \dfrac{a}{b^2+1}$
\end{chapex}

\begin{chapex}
$a \odot b = \dfrac{a}{\sqrt{b^2+1}}$
\end{chapex}

\begin{chapex}
\label{cqClosureOfNumberSetsEnd}
$a \odot b = \begin{cases} a^{b} & \text{if $b > 0$} \\ 0 & \text{if $b \not\in \mathbb{Q}$} \end{cases}$
\end{chapex}
\end{multicols}

\begin{definition}
A complex number $\alpha$ is \textbf{algebraic} if $p(\alpha) = 0$ for some nonzero polynomial $p(x)$ over $\mathbb{Q}$.
\end{definition}

\begin{chapex}
Let $x$ be a rational number. Prove that $x$ is an algebraic number.
\end{chapex}

\begin{chapex}
Prove that $\sqrt{2}$ is an algebraic number.
\end{chapex}

\begin{chapex}
Prove that $\sqrt{2} + \sqrt{3}$ is an algebraic number.
\end{chapex}

\begin{chapex}
Prove that $x+yi$ is an algebraic number, where $x$ and $y$ are any two rational numbers.
\end{chapex}

\subsection*{True--False questions}

\tfquestiontext{cqGettingStartedTFBegin}{cqGettingStartedTFEnd}

\begin{chapex} % False
\label{cqGettingStartedTFBegin}
Every integer is a natural number.
\end{chapex}

\begin{chapex} % True
Every integer is a rational number.
\end{chapex}

\begin{chapex} % True
Every integer divides zero.
\end{chapex}

\begin{chapex} % True
Every integer divides its square.
\end{chapex}

\begin{chapex} % True
The square of every rational number is a rational number.
\end{chapex}

\begin{chapex} % False
The square root of every positive rational number is a rational number.
\end{chapex}

\begin{chapex} % False
When an integer $a$ is divided by a positive integer $b$, the remainder is always less than $a$.
\end{chapex}

\begin{chapex} % False
\label{cqGettingStartedTFEnd}
Every quadratic polynomial has two distinct complex roots.
\end{chapex}

\subsection*{Always--Sometimes--Never questions}

\asnquestiontext{cqGettingStartedASNBegin}{cqGettingStartedASNEnd}

\begin{chapex} % Always
\label{cqGettingStartedASNBegin}
Let $n,b_1,b_2 \in \mathbb{N}$ with $1 < n < b_1 < b_2$. Then the base-$b_1$ expansion of $n$ is equal to the base-$b_2$ expansion of $n$.
\end{chapex}

\begin{chapex} % Never
Let $n,b_1,b_2 \in \mathbb{N}$ with $1 < b_1 < b_2 < n$. Then the base-$b_1$ expansion of $n$ is equal to the base-$b_2$ expansion of $n$.
\end{chapex}

\begin{chapex} % Sometimes
Let $a,b,c \in \mathbb{Z}$ and suppose that $a$ divides $c$ and $b$ divides $c$. Then $ab$ divides $c$.
\end{chapex}

\begin{chapex} % Always
Let $a,b,c \in \mathbb{Z}$ and suppose that $a$ divides $c$ and $b$ divides $c$. Then $ab$ divides $c^2$.
\end{chapex}

\begin{chapex} % Always
Let $x,y \in \mathbb{Q}$ and let $a,b,c,d \in \mathbb{Z}$ with $cy+d \ne 0$. Then $\dfrac{ax+b}{cx+d} \in \mathbb{Q}$.
\end{chapex}

\begin{chapex} % Sometimes
Let $\dfrac{a}{b}$ be a rational number. Then $a \in \mathbb{Z}$ and $b \in \mathbb{Z}$.
\end{chapex}

\begin{chapex} % Sometimes
Let $x \in \mathbb{R}$ and assume that $x^2 \in \mathbb{Q}$. Then $x \in \mathbb{Q}$.
\end{chapex}

\begin{chapex} % Always
Let $x \in \mathbb{R}$ and assume that $x^2+1 \in \mathbb{Q}$ and $x^5+1 \in \mathbb{Q}$. Then $x \in \mathbb{Q}$.
\end{chapex}

\begin{chapex} % Always
\label{cqGettingStartedASNEnd}
Let $p(x) = ax^2+bx+c$ be a polynomial with $a,b,c \in \mathbb{R}$ and $a \ne 0$, and suppose that $u+vi$ be a complex root of $p(x)$ with $v \ne 0$. Then $u-vi$ is a root of $p(x)$.
\end{chapex}