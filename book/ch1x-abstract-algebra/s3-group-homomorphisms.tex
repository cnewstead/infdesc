\section{Groups}
\secbegin{secGroups}

\todo{}

\begin{definition}
\label{defOrderOfElementOfGroup}
\index{order!of an element of a group}
\nindex{order}{$o(a)$}{order of an element of a subgroup}
Let $G$ be a group and let $a \in G$. If $a^k = e$ for some $k \ge 1$, then the \textbf{order} of $a$ is the natural number $o(a)$ defined by
\[ o(a) = \mathrm{min} \{ k \ge 1 \mid a^k = e \} \]
If $a^k \ne e$ for any $k \ge 1$, we say that $a$ has \textbf{infinite order} and write $o(a) = \infty$.
\end{definition}

\todo{}

\begin{definition}
\label{defSubgroup}
\index{subgroup}
Let $G$ be a group. A \textbf{subgroup} of $G$ is a subset $H \subseteq G$, such that $H$ inherits a group structure from the group structure of $G$---that is, such that
\begin{enumerate}[(i)]
\item $e \in H$;
\item $ab \in H$ for all $a,b \in H$; and
\item $a^{-1} \in H$ for all $a \in H$.
\end{enumerate}
We write $H \le G$ \inlatex{le} to denote the assertion that $H$ is a subgroup of $G$.
\end{definition}

\begin{exercise}
Let $G$ be a group and let $H \subseteq G$ be a subset. Prove that if $ab^{-1} \in H$ for all $a,b \in H$, then $H$ is a subgroup of $G$.
\hintlabel{exCriterionForSubgroup}{%
You need to verify that $e \in H$, that $a^{-1} \in H$ for all $a \in H$, and that $ab \in H$ for all $a,b \in H$. Prove these facts in this order.
}
\end{exercise}

\begin{strategy}[Proving a subset of a group is a subgroup]
\label{strCriterionForSubgroup}
Let $G$ be a group and let $H \subseteq G$ be a subset. In order to prove that $H$ is a subgroup of $G$, it suffices to prove that $ab^{-1} \in H$ for all $a,b \in H$.
\end{strategy}

\todo{}

\subsection*{Cosets}

\begin{definition}
\label{defCoset}
\index{coset}
Let $G$ be a group and let $H \le G$. Given $a \in G$, the \textbf{left coset} of $H$ with respect to $a$ is the subset $aH \subseteq G$ defined by
\[ aH = \{ ah \mid h \in H \} \]
and the \textbf{right coset} of $H$ with respect to $a$ is the subset $Ha \subseteq G$ defined by
\[ Ha = \{ ha \mid h \in H \} \]
Write $G / H = \{ gH \mid g \in G \}$ for the set of all left cosets of $H$ and $H \setminus G = \{ aH \mid a \in G \}$ for the set of all right cosets of $H$.
\end{definition}

\todo{}

\begin{exercise}
\label{exCosetEquivalenceRelation}
Let $G$ be a group and let $H \le G$. Define a relation $\sim_H$ on $G$ by
\[\forall a,b \in G,~ a \sim_H b \Leftrightarrow ab^{-1} \in H \]
Prove that $\sim_H$ is an equivalence relation on $G$ and that $[a]_{\sim_H} = aH$ for all $a \in G$, and deduce that $G/{\sim_H} = G/H$.
\end{exercise}

\begin{strategy}[Proving two cosets are equal]
\label{strProvingCosetsAreEqual}
Let $G$ be a group and let $H \le G$. Given $a,b \in G$, in order to prove $aH = bH$, it suffices to prove $ab^{-1} \in H$.
\end{strategy}

\begin{lemma}
\label{lemCosetsHaveEqualSize}
Let $G$ be a finite group and let $H \le G$. Then $|aH| = |bH|$ for all $a,b \in G$.
\end{lemma}

\begin{cproof}
Define a function $f : aH \to bH$ by $f(x) = ba^{-1}x$ for all $x \in aH$. Then
\begin{itemize}
\item \textbf{$f$ is well-defined.} To see this, let $x \in aH$. Then $x=ah$ for some $h \in H$, and so $f(x) = ba^{-1}ah = bh$. So $f(x) \in bH$, as claimed.
\item \textbf{$f$ is bijective.} To see this, note that the function $g : bH \to aH$ defined by $g(y) = ab^{-1}y$ for all $y \in bH$ is an inverse for $f$. Indeed, $g$ is well-defined for the same reasons as $f$, and
\[ f(g(x)) = ba^{-1}ab^{-1}x = bb^{-1}x = x \quad \text{and} \quad g(f(y)) = ab^{-1}ba^{-1}y = aa^{-1}y = y\]
for all $x \in aH$ and $y \in bH$. So $g$ is an inverse for $f$.
\end{itemize}

Since $f : aH \to bH$ is a bijection, we have $|aH| = |bH|$ by \Cref{strBijectiveProof}.
\end{cproof}

\todo{}

\begin{definition}
\label{defIndexOfSubgroup}
\index{index}
Let $G$ be a finite group and let $H \le G$. The \textbf{index} of $H$ in $G$ is the natural number $[G:H]$ defined by $[G:H] = |G/H|$.
\end{definition}

\todo{}

\begin{theorem}[Lagrange's theorem]
\label{thmLagrange}
Let $G$ be a finite group and let $H \le G$. Then $|G| = [G : H] |H|$.
\end{theorem}

\begin{cproof}
By \Cref{exCosetEquivalenceRelation}, we know that $G/H$ is a quotient of $G$ by an equivalence relation, so that the cosets of $H$ in $G$ partition $G$ by \Cref{exQuotientIsPartition}. By \Cref{lemCosetsHaveEqualSize}, we have $|aH| = |bH|$ for all $a,b \in G$---in particular, $|aH| = |eH| = |H|$ for all $a \in H$.

By the addition principle (\Cref{strAdditionPrinciple}) we have
\[ |G| = \sum_{aH \in G/H} |aH| = \sum_{aH \in G/H} |H| = |G/H| |H| = [G:H] |H| \]
as required.
\end{cproof}

Lagrange's theorem has an enormous number of useful consequences. Here is one of them.

\begin{corollary}
\label{corSizeOfSubgroupDividesSizeOfGroup}
Let $G$ be a finite group. The size of every subgroup of $G$ divides the size of $G$.
\end{corollary}

\begin{cproof}
Let $H \le G$. Then $|G| = [G:H] |H|$ by Lagrange's theorem, and so $|H|$ divides $|G|$ by \Cref{defDivision}.
\end{cproof}

\todo{}

\begin{exercise}
Let $G$ be a finite group and let $g \in G$. Prove that $o(g)$ divides $|G|$.
\end{exercise}

\todo{}